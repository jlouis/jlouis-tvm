\documentclass[a4paper, oneside, 10pt, draft]{memoir}
\input{packages}
\chapterstyle{culver}

\usepackage{fixme}
\usepackage[english]{babel}
%\usepackage[utopia]{mathdesign}

\usepackage[osf,sc]{mathpazo}
\linespread{1.05}
%\usepackage{fourier}

\usepackage{semantic}
\author{Jesper Louis
  Andersen\\jesper.louis.andersen@gmail.com\\140280-2029}
\title{Formalizing a Virtual Machine}
\date{\today}

\newlength{\drop}
\newcommand*{\titleM}{\begingroup% Misericords, T&H p 153
  \drop = 0.08\textheight
  \centering
  {\Huge\bfseries Lambda}\\[\baselineskip]
  {\scshape IR of exsml}\\[\baselineskip]
  {\scshape by}\\[\baselineskip]
  {\large\scshape Jesper Louis Andersen\\jesper.louis.andersen@gmail.com}\par
  \endgroup}

\bibliographystyle{plain}

\begin{document}
\maketitle{}
\tableofcontents{}
\chapter{Introduction}

The Low-Level Virtual Machine \cite{llvm:homepage} is a framework for
constructing programming language tool-chains. It provisions for
building JIT compilers, or traditional static compilations. LLVM
provides a low-level language, much like an idealized typed assembler
as an API for compiler writers to target. This language is an
Intermediate Language (IL) from which the LLVM backend can emit code,
optimize programs and the like.

The goal of LLVM is to make the work of compiler writers easier. It
provides a stepping stone as soon as you can target its IL. The idea
is then that many optimizations come ``for free'' in the sense that
the LLVM backend is able to carry them out.

Unfortunately, the description of the LLVM IL is completely
informal. The main reference document \cite{llvm:reference} contains
next to no specification in any formal language. There is a complete
lack of (E)BNF grammar syntax, semantics and type system.

This report establishes a preliminary syntax and semantics for LLVM
from which one can build a complete work.

\chapter{Design considerations}

%% Mention that SSA is the key. Build it up from an example and make
%% it clear that the SSA-trick can be utilized to lift the problem.

\bibliography{biblio}
% State what the goals of this project is.
% State we focus on the operational semantics.
% State we will use Twelf where applicable if time allows.

\end{document}

%%% Local Variables: 
%%% mode: latex
%%% TeX-master: t
%%% End: 
