\documentclass[a4paper, oneside, 10pt, draft]{memoir}
%% This file is my default inclusion of packages I need when writing
%% things in LaTeX. The list here is pretty comprehensive, but sets up
%% the main stuff I use in all projects I start on.

%% This part sets up the font basics. I like mathpazo a lot. It sets
%% up Palatino as a clone with pazo math. It is much more beautiful
%% than many other fonts you can choose from. I also like some of the
%% mathdesign variants, in particular [utopia,osf]{mathdesign} with no
%% linespread. We could make it a setting here.
%%
%% Also, add microtype. We can't really live without it.
\usepackage{microtype}
\usepackage[T1]{fontenc}
\usepackage{lmodern}
%\usepackage[osf]{mathpazo}
%\linespread{1.05}
%\usepackage[sc,osf,utopia]{mathdesign}

%% AMS gives us the basic math-support
\usepackage{amsmath}

%% AMS Theorem support. We set up a good list of theorems and make
%% sure that they are numbered appropriately in the order of main
%% theorems. I hate when a document contains too many counters so this
%% keeps the number of live counters down to a minimum.
\usepackage{amsthm}
\theoremstyle{plain}
\newtheorem{axm}{Axiom}
\newtheorem{thm}{Theorem}
\newtheorem{lem}[thm]{Lemma}
\newtheorem{prop}[thm]{Proposition}
\newtheorem*{cor}{Corollary}

\theoremstyle{definition}
\newtheorem{defn}[thm]{Definition}
\newtheorem{conj}[thm]{Conjecture}
\newtheorem{exmp}[thm]{Example}
\theoremstyle{remark}
\newtheorem*{rem}{Remark}
\newtheorem*{note}{Note}
\newtheorem{case}{Case}


%% Hyperref should be last
\usepackage{hyperref}

\chapterstyle{culver}

\usepackage{fixme}
\usepackage[english]{babel}
%\usepackage[utopia]{mathdesign}

\usepackage[osf,sc]{mathpazo}
\linespread{1.05}
%\usepackage{fourier}

\usepackage{semantic}
\author{Jesper Louis
  Andersen\\jesper.louis.andersen@gmail.com\\140280-2029}
\title{Formalizing Assembly Language}
\date{\today}

\newlength{\drop}
\newcommand*{\titleM}{\begingroup% Misericords, T&H p 153
  \drop = 0.08\textheight
  \centering
  {\Huge\bfseries Lambda}\\[\baselineskip]
  {\scshape IR of exsml}\\[\baselineskip]
  {\scshape by}\\[\baselineskip]
  {\large\scshape Jesper Louis Andersen\\jesper.louis.andersen@gmail.com}\par
  \endgroup}

\bibliographystyle{plain}

\begin{document}
\maketitle{}
\chapter*{Problem statement}

Computers understands only machine code. While easy for computers,
machine code is rather hard for humans to write programs in. The
solution is then to write the program in a high-level language and
then \emph{translate} the program by a compiler to a target
language. In the recent years, it has been popular to target not the
machine code of a CPU but an abstract \emph{Virtual Machine}.

One such virtual machine is LLVM, the Low-level virtual machine. It is
named as such since it provides a language which is almost in
isomorphism with machine code, thus low-level. The LLVM language
called the IR\footnote{A shorthand for Intermediate Representation}
is, essentially, a typed assembly language with an
SSA-form\cite{appel:modern} requirement.

However, the LLVM IR has no formal semantics. It is specified by an
informal text-description. In addition, it is not clear that the type
system for the IR has the property of type safety\cite{pierce:types}.

Our thesis is:

``It is feasible and useful to produce an operational semantics for a
subset of the LLVM IR''

Our formalization will probably digress and not follow the LLVM IR
blindly. Rather, we will let us inspire by the constructions in the IR
but diverge where formalization will be too hard. Also we stress that
we will analyze a subset of the IR. The full IR is too time-consuming
to formalize and we do not know if the type system is type safe. We
aim to produce an operational semantics which \emph{is} type-safe,
even if this fragment is rather small.

An operational semantics of the fragment is useful in the sense that
formalizations often gives deeper insight than informal
descriptions. We hope to gain some additional insights in the
formalization work.

To embrace the state-of-the-art of formalization, we plan we verify
correctness by translating the semantics, and if time allows key
proofs, into the Twelf Logical Framework. This will enable us to
verify correctness by machine.

At the project end, we will have produced:
\begin{itemize}
\item A typed assembly language (TAL), inspired by the rules of the LLVM
  IR. This TAL will be given an operational semantics.
\item An encoding of key parts of the semantics in Twelf; if time
  allows, everything.
\item A report describing the product.
\end{itemize}

\paragraph{Related work}

Typed assembly languages have seen much work in the litterature, see
\cite{crary:..}, \cite{morisett:...}. \fixme{Cite Crary, Morisett,
  et. el}. There are even formalizations of some of these languages in
Twelf. This allows us to use earlier experiences in working with
assembly-style language. To our knowledge
however, there has been no attempts at formalizing LLVM in Twelf.

\chapter*{Learning goals}

In this project, our primary goal is to become better at taking
informal descriptions and formalizing them into logical systems. We
note that no course at DIKU can teach you this skill currently as they
all provide some skeleton-semantics for the students to work on. In
this project, we begin with a black slate and inspiration in the
existing litterature on the subject of typed assembly languages.

Our second goal is the become better at working with the Twelf logical
framework. In particular, we plan on constructing a bigger development
than the Twelf toy-examples from the introductory course ``Computation
and Deduction''. Getting proofs into a form which can be verified by
the machine can be quite challenging, but we reap the benefits of
added assurance on proof correctness.

Finally, it is a goal to hone our report writing skills and improve
our academic prose.

\bibliography{biblio}
% State what the goals of this project is.
% State we focus on the operational semantics.
% State we will use Twelf where applicable if time allows.

\end{document}
